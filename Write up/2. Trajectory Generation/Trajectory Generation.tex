\documentclass[11pt]{article}
\usepackage{geometry}
\geometry{a4paper, left=37.5mm, right=20mm, top=20mm, bottom=20mm} %sets margins to correct size
\usepackage{setspace}
\usepackage[super]{nth}

\title{Trajectory Generation}
\date{\vspace{-5ex}} %ommits date, and date spacing.

\begin{document}
\maketitle
\onehalfspacing
As described in section 1.2.2, in the current version of the MyPAM the game generates equidistant intermediate positions between the start position and the final position of each reaching movement and passes these one at a time to the low-level controller. This leads to a number of issues:
%Start numbered list
	\begin{enumerate}
	\item A linear trajectory of this nature is not 		reflective of natural human motion, although
	there is no evidence that normative trajectories 		which mimic human motion promote
	neurofunctional plasticity (Marchal-Crespo and 			Reinkensmeyer, 2009).
	\item The game does not operate at 30 Hz reliably 		as a result of being dependant of the
	non-deterministic Microsoft Windows Operating 			System (OS). There may be
	instances where the game rate will drop, resulting 		in incorrect intermediate position
	data being sent to the controller.
	\item There are occasions where no intermediate 		points are generated and the final target
	position is sent to the controller as the next 			target, for example during some game
	types and during transitions between different 			games. This leads to a large difference
	between the current position and the target 			position, and large motor demands are
	generated. This leads to aggressive accelerations 		and potentially dangerous
	interaction forces between the patient and the 			robot.
	\end{enumerate} 
%End numbered list
\section{Generating a Smooth Trajectory}
A smooth trajectory is desirable when assisting the user to reach a target. This is
because smooth trajectories are necessary to ensure safe interaction between the robot and
the patient (Amirabdollahian et al, 2002) and movement smoothness is an indicator of
increased motor control after stroke according to Balasubramanian et al (2015), though there
is no evidence to suggest that assisting a smooth movement increases neurofunctional
plasticity. Mathematically, a smooth trajectory translates to minimising the rate of change of
an input, where the input corresponds to the order of the system. For example, a \nth{1} order
system corresponds to a kinematic model where velocities may be arbitrarily specified. This
is summarised in the table 3.1 below:\\
\begin{table}
	\centering
	%\begin{center}
	\begin{tabular}{|c|c|}
	\hline
	\textbf{Order of the system} &\textbf{Input to the 		system}\\ \hline
	\nth{1} &10 \\ \hline
	\nth{2} &10 \\ \hline
	\nth{3} &10 \\ \hline
	\nth{4} &10 \\ \hline
	\nth{5} &10 \\ \hline
	\nth{6} &10 \\ \hline
	\end{tabular}
	\caption{Table to test captions and labels}
	\label{table:1}
\end{table}

\end{document}
